\section{Related Work}
% discusstion
% - model is still quite small. cam we use bigger CNNs?
\fixme{What we want to say:
  1) We showed DNN is feasible on today's embedded computing
  platforms.
  2) But, the CNN model is rather smaller-kind with respect to today's
  state-of-the-art CNNs, which are much bigger.
  3) disscuss google's object detection api
  paper~\cite{huang2017speed}. 
  4) Applying these much more complex DNNs (such as google's detection
  models) may be challenging for embedded computing
  platforms.
  5) Two general approaches: (1) make it faster (computing workload
  remains the same) (2) reduce the workload itself (model compression
  or reduced precision etc.)}
  
While we were able to show that a smaller embedded platform like the 
Raspberry Pi 3 can successfully run a DNN, it should be noted that the
network architecture, and hence model, used was relatively small. Other 
available DNN's can be much larger compared to the architecture of the
DAVE-2. Such DNN's can be seen in Google's Object Detection API~\cite{huang2017speed}.
The smallest model available in the API is the MobileNet model, which
has 3,191,072 parameters and is approximately 12.7 larger than the
model used by the DeepPicar. The other models all have parameter sizes 
of at least 10 million. Due to the considerable differences in the 
model sizes, it is likely that running them on platforms like the Pi 3,
and still get acceptable real-time performance would be difficult, 
if not infeasible. 

% related work on efficient DNN representation and processing.
While repeated feedforward and backpropagation operations during
the training time account for most of the computational cost in
deep learning, there is a growing need for an improved efficiency
during the inferencing time as well, especially because of the
potential of utilizing Deep Neural Networks (DNN) for the real-time
pattern recognition tasks in embedded systems, as our paper
exemplifies.

When a DNN is deployed in those implementations with limited
resources, such as memory and power, the floating-point operations
involved in the large matrix multiplications are a burdensome task.
Many recent advances in network compression showed promising results
in reducing such computational cost during the feedforward
process. The fundamental assumption in those techniques is that the
DNNs are redundant in their structure and representation. For example,
network pruning can thin out the network and provides a more condensed
topology~\cite{han2015deep}.
Another common compression method is to reduce the
quantization level of the network parameters, so that arithmetic
defined with the floating-point operations are replaced with low-bit
fixed-point counterparts. To this end, single bit quantization of the
network parameters or ternary quantization have been recently proposed
~\cite{hwang2014fixed,soudry2014expectation,kim2016bitwise,rastegari2016xnor,hubara2016binarized,beauchamp2006embedded,govindu2004analysis}.
In those networks, the inner product between the
originally real-valued parameter vectors is defined with XNOR followed
by bit counting, so that the network can greatly minimize the
computational cost in the hardware implementations. This drastic
quantization can produce some additional performance loss, but those
new binarized or ternarized systems provide a simple quantization
noise injection mechanism during training so that the additional error
is minimized to an acceptable level.

The XNOR operation and bit counting have been known to be efficient in
hardware implementations. In~\cite{rastegari2016xnor}, it was shown
that the binarized convolution can substitute the expensive
convolutional feedforward operations in a regular Convolutional Neural
Network (CNN), %% by using only about 1.5\% of the memory space, while
providing 20 to 60 times faster feedforward. Binary weights were also
able to provide 7 times faster feedforward than a floating-point
network for the hand written digit recognition task as well as 23
times faster matrix multiplication tasks on
GPU~\cite{hubara2016binarized}.
Moreover, FPGA implementations showed that the XNOR operation is 200
times cheaper than floating-point multiplications with 
single precision~\cite{beauchamp2006embedded,govindu2004analysis}.
XNOR-POP is another hardware implementation that reduced
the energy consumption of a CNN by 98.7\%~\cite{jiang2017xnor}.

% related work on testbeds
\fixme{maybe we can also discuss MIT racecar, UPENN car, donkey car and other testbeds?
  The points of discussion may include: 1) they are expensive. 2)
  donkey car is similar to us, but still a bit more expnsive, and more
  importantly there is no existing studies which systematically evaluate
  real-time performance of using DNN on embedded computing
  platforms. Our work provides quantitive real-time performance
  evaluation results, using a known DNN used in a real self-driving
  car. Our work is fully reproducible at a very low cost, lowring the
  barrier to entry in researching self-driving cars, helping the
  community blah blah blah.}
  
Another autonomous vehicle platform is the Donkey Car\footnote{https://donkeycar.com}.
Like the DeepPicar, the Donkey Car is a low cost platform that can be 
utilized for researching self-driving cars. However, comparatively, the
DeepPicar is approximately \$200 cheaper than the Donkey Car. 
Furthermore, the Donkey Car was created with an emphasis on model accuracy
and the ability of the car to navigate the desired environment. We are 
more focused on the real-time performance of the DeepPicar, and whether
the Raspberry Pi 3 embedded computing platform can properly support autonomous 
vehicle operations. To the best of our knowledge, no other studies have
been done to evaluate the capabilities of the Pi 3 for running DNNs and
performing autonomous vehicle operations.
