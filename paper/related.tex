\section{Related Work}\label{sec:related}

%  NN for autonomous systems
\cite{Bojarski2016}

% open testbed
\cite{Kato:2015}
~\cite{shin2017project}
~\cite{upennf1tenth}

% real-time performance evaluation of embedded computing platforms
~\cite{Otterness2017} vision workload
\cite{Amert2017} TX2 scheduling algorithm
\cite{NVIDIA2015} dnn workload real-time inference

% acceleration
\cite{Jouppi2017} TPU

Autonomous vehicles are becoming an increasingly interesting research
topic as computing platforms capable of safely processing sensor input
into safe vehicle controls become smaller and more affordable.  The
model proposed in End-To-End Learning for Self-Driving Cars
\cite{Bojarski2016} is viable on a platform as simple as a Raspberry
Pi. 

Given smaller, affordable platforms with multiple cores and even GPU
based parallel processing such as the Tegra X1 \cite{NVIDIA2015} and
the recently released Tegra X2 \cite{Amert2017}, allocation and
management of available shared CPU and GPU resources \cite{Kim2016}
becomes increasingly important.  Research on improving the performance
of neural networks \cite{Jouppi2017} also attributes to the viability
of using these models in small systems.  With less powerful onboard
computing devices, it was necessary for these kinds of robotics
systems to use an external computer for the algorithm such as is used
in \cite{LeCun:2005}. 

Together with hardware and platform improvements, there have been
significant improvements in the algorithms such as a neural network
algorithm capable of surpassing human level control at Atari games
\cite{DBLP}.  A popular technique had been to take sensor input and
determine a depth map, such as the model used in
\cite{Michels:2005}. End-to-end models such as the one used in
\cite{Bojarski2016} can improve the efficiency of the model.
Improving the flexibility of the model and it ability to adapt to
different situations and avoiding overfitting to training data
continues to be an important topic in autonomous driving
\cite{Pomerleau1989}. 

Producing large platforms such as DAVE \cite{LeCun:04} has often been
the approach to developing the technology for self-driving vehicles.
This approach, while certainly effective, presents a higher barrier to
entry for research groups.  Recently, there has been a movement to
make research towards autonomous driving more accessible.
\cite{Kato:2015} have provided an open platform for developing
autonomous vehicles using publicly available vehicles and sensors.
Smaller, more focused platforms, such as the one found in
\cite{Michels:2005} also present an accessible way forward.  A current
accesible open-source autonomous platform is the Donkey Car which is
available for approximately \$200\footnote{http://www.donkeycar.com/}.
These developments in research have made small affordable vehicles with
onboard processing easily accessible.
