\section{Conclusion}\label{sec:conclusion}
We presented DeepPicar, a low cost autonomous car platform that is
inexpensive to build, but is based on state-of-the-art AI technology:
End-to-end deep learning based real-time control.
Specifically, DeepPicar uses a deep convolutional neural network to
predict steering angles of the car directly from camera input data
in real-time. Importantly, DeepPicar's neural network architecture is
identical to that of NVIDIA's real self-driving car. 

Despite the complexity of the neural network, DeepPicar uses a
low-cost Raspberry Pi 3 quad-core computer as its sole computing
resource. We systematically analyzed the real-time characteristics of
the Pi 3 platform in the context of deep-learning based real-time
control appilcations, with a special emphasis on real-time deep neural
network inferencing.
We also evaluated other, more powerful, embedded computing
platforms to better understand achievable real-time performance of
DeepPicar's deep-learning based control system and the impact of
computing hardware architectures.
We find all tested embedded platforms, including the Pi 3, are capable
of supporting deep-learning based real-time control, from 20 Hz up to
100 Hz, depending on the platform and its system
configuration. However, shared resource contention remains an
important issue that must be considered in applying deep-learning
models on shared memory based embedded computing platforms.

As future work, we plan to apply shared resource management
techniques~\cite{Yun2013,yun2014rtas} on the DeepPicar platform and
evaluate their impact on the real-time performance of the system. We
also plan to improve the prediction accuracy by feeding more data and
upgrading the RC car hardware platform to enable more precise steering
angle control.