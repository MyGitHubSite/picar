\section{Conclusion}\label{sec:conclusion}
We presented DeepPicar, a low cost autonomous car platform that is
inexpensive to build, but is based on state-of-the-art AI technology:
End-to-end deep learning based real-time control.
Specifically, DeepPicar uses a deep convolutional neural network to
predict steering angles of the car directly from camera input data
in real-time. Importantly, DeepPicar's neural network architecture is
identical to that of NVIDIA's real self-driving car DAVE-2. 

Despite the complexity of the neural network, DeepPicar uses a
low-cost embedded quad-core computer, the Raspberry Pi 3,  as its sole
computing resource.
We systematically analyzed the platform's real-time capability in
supporting the CNN-based real-time control task.
We also evaluated other, more powerful, embedded computing
platforms to better understand achievable real-time performance of
DeepPicar's CNN based control system and the impact of
computing hardware architectures.
We find all tested embedded platforms, including the Pi 3, are capable
of supporting the CNN based real-time control, from 20 Hz up to
100 Hz, depending on the platform.
Futhermore, all platforms were capable of consolidating
multiple CNN models and/or tasks.

However, we also find that shared resource
contention remains an important issue that must be considered to
ensure desired real-time performance on these shared memory based
embedded computing platforms.
Toward this end, we evaluated the impact of shared resource contention
to the CNN workload in diverse consolidated workload setups, and
evaluated the effectivness of state-of-the-art shared resource
isolation mechanisms in protecting performance of the CNN
based real-time control workload.

%% As future work, we plan to improve the prediction accuracy of the
%% system by feeding more data in training.
As future work, we plan to investigate ways to reduce computational
and memory overhead of CNN inferencing and to evaluate the
effectiveness of FPGA and other specialized accelerators.

%% we plan to apply shared DRAM resource management
%% techniques~\cite{Yun2013,yun2014rtas} on the DeepPicar platform and
%%evaluate their impact on the real-time performance of the system. 

%% and
%% upgrading the RC car hardware platform to enable more precise steering
%% angle control
