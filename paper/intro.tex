\section{Introduction} \label{sec:intro}

Autonomous vehicles have been a topic of increasing interest in many
disciplines in recent years. Due to the nature of autonomous vehicles,
it is necessary that all real-time operations are successfully
performed prior to their deadlines. This requires that the AV platform
be capable of completing all necessary computations in a timely
manner, while maintaining a high level of accuracy. Consequently, the
cost of creating a platform capable of self-driving can be relatively
high, especially in the current cost-sensitive climate of the
automotive industry \cite{}. The overall price for an autonomous
vehicle platform currently acts as a bottleneck, especially from an
academic standpoint, where cost is an important factor. For many
researchers and students, the charge associated with autonomous
vehicle research is a significant barrier. These same parties may also
be deterred from participating in autonomous vehicle research due to
the potential for their developed platforms to break in the
experimental process. The idea of building one relatively expensive
platform may not be too daunting for some, but the notion of
potentially making multiple identical platforms in case of accidents
may be overwhelming. Toward this end, we explore the possibility of
developing a low-cost system that still employs state-of-the-art AI
technologies and is capable of executing real-time autonomous vehicle
operations. 
	
In this paper, we present DeepPicar, a low-cost autonomous car
platform for research and education. From hardware perspective, 
DeepPicar is comprised of a Raspberry Pi 3 Model B quad-core
computer, a web camera and a RC car, all of which are affordable
components (less than \$200 in total).
The DeepPicar, however, employs state-of-the-art AI
techonologies, including a vision-based end-to-end control system that
utilizes a deep convolutional nueral network (CNN).
The network receives an image frame from a single forward
looking camera, as input and generates a predicted steering angle
value as output at each control period in \emph{real-time}. 
The network is deep as it has 9 layers, about 27 million connections
and 250 thousand parameters (weights).
The network architecture is in fact identifical to the one
used in Nvidia's DAVE-2 self-driving car~\cite{Bojarski2016}. Other
than the obvious difference in scale (RC car vs. real car), from the
computing perspective, the only differences between
the two systems are that our system is implemented in
TensorFlow~\cite{abadi2016tensorflow} and runs on a 
Raspberry Pi 3 whereas Nvidia's DAVE-2 systems is implemented in Torch
7~\cite{collobert2011torch7} and runs on a Drive PX computer (NVIDIA's
automotive specialized computing system~\cite{drivepx}), which is more
powerful but also more expensive.

We follow the standard supervised learning methodology to train the
system with a goal of following lanes on the ground, mimicking real
roads. We first collected data for taining and validation by manually 
controlling the RC car and recording the vision (from the webcam
mounted on the RC-car) and the human control inputs. We then train the
network offline using the collected data on a desktop computer, which
equips a Nvidia GTX 1060 GPU. Finally, the trained network is copied
back to the Raspberry Pi, which is then used to perform interference
operations---locally on the Pi---in the car's main control loop in
real-time. For real-time control, each \emph{inference} operation must
be completed within the desired control period. (e.g., 50ms period for
20Hz control frequency.)

The {\bf contributions} of this paper are as follows:
\begin{itemize}
  \item We present the design and implementation of a
    low-cost autonomous vehicle testbed DeepPicar, which utilizes the
    state-of-the-art aritificial intelligence techniques.
  \item We provide an analysis and case-study of real-time issues in the
    DeepPicar.
  \item We systematically compare real-time computing capabilities of
    multiple embedded computing platforms in the contex of a
    vision-based autnomous driving.
\end{itemize}

%% We chose to use a Raspberry Pi 3 not only because of its
%% affordability and availability but also because it is representative
%% of today's mainstream low-end multicore processors, found in
%% smartphones, tables and other embedded devices.
%% The DeepPicar is comprised of a Raspberry
%% Pi 3 Model B, a camera and a low cost RC car.

%% By creating and using the Picar platform, we seek to accomplish the 
%% following three goals. First, the significant reduction of the overall
%% cost required 
%% would make autonomous driving research more accessible to interested
%% individuals/parties. Furthermore, it would remove the concerns of
%% creating replacement platforms as each part, or the entire system,
%% could be replaced relatively inexpensively. Second, the utilization of
%% a more cost-efficient platform would allow for a better understanding
%% of the performance requirements necessary for efficiently operating
%% autonomous vehicles. Finally, we strive to achieve the ability to
%% compare the real-time performance of multiple embedded computing
%% platforms. 

%% We display the efficacy of the Picar by training and testing the
%% DeepTesla DNN \cite{} in a custom-made environment. We find that the
%% Picar is capable of consistently accomplishing all real-time
%% operations required for autonomous driving in a 50 millisecond frame. 

The remaining sections of the paper are as follows: Section II gives
an overview of the platform, including the high-level system and the
methods used for training and inference. Section III discusses the
ways/methodologies in which training data was collected. Section IV
outlines the network architecture. This is followed by the real-time
DNN inferencing in Section V. Section VI reviews the online/offline
training done with the platform, with an evaluation given in Section
VII. Section VIII gives a discussion of related works, and the paper
finishes with conclusions in Section IX. 
