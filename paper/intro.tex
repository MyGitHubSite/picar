\section{Introduction} \label{sec:intro}

% broad context:
% - advance in ai sparked interests in the robotics application, such as
%   self-driving cars.
% - in particular, deep neural network models are increasingly used
%   for perception and control of a vehicle. say. AI workloads.
%
%
Autonomous cars have been a topic of increasing interest in recent
years as many companies are actively developing related hardware
and software technologies toward fully autonomous driving capability with
no human intervention. Deep neural networks (DNNs) have been
successfully applied in various perception and control tasks in
recent years.  They are important workloads for autonomous vehicles
as well. For example, Tesla Model S was known to use a specialized
chip (MobileEye EyeQ), which used a deep neural network for vision-based
real-time obstacle detection and avoidance. More recently, researchers
are investigating DNN based end-to-end control of
cars~\cite{Bojarski2016} and other robots. It is expected that more
DNN based Artificial Intelligence workloads may be used in future
autonomous vehicles.

% big problem
Executing these AI workloads on an embedded computing platform 
poses several additional challenges. First, many AI workloads in vehicles 
are computationally demanding and have strict real-time requirements. 
For example, latency in a vision-based object
detection task may be directly linked to the safety of the vehicle. This
requires a high computing capacity as well as the means to guaranteeing
the timings. On the other hand, the computing hardware platform must
also satisfy cost, size, weight, and power constraints, which require a
highly efficient computing platform. These two conflicting
requirements  complicate the platform selection process as observed in
~\cite{Otterness2017}.

%% For example, while today's self-driving car
%% prototype equip more \$100,000
%% of computers and sensors~\cite{juliussen2014emerging}, a study
%% found that aveage consumers are willing to pay much less amount of
%% extra cost for a self-driving capability~\cite{Daziano2017}.
%% https://qz.com/924212/what-it-really-costs-to-turn-a-car-into-a-self-driving-vehicle/

% related work and remaining problems
To understand what kind of computing hardware is needed for AI
workloads, we need a testbed and realistic workloads. While using a real 
car-based testbed would be most ideal, it is not only highly expensive, but also
poses serious safety concerns that hinder development and exploration.
Therefore, there is a strong need for safer and less costly
testbeds. There are already several relatively inexpensive RC-car
based testbeds, such as MIT's 
RaceCar~\cite{shin2017project} and UPenn's F1$/$10 racecar~\cite{upennf1tenth}.
However, these RC-car testbeds still cost more than \$3,000, requiring
considerable investment.

% our goals
Instead, we want to build a low cost testbed that still employs the
state-of-the art AI technologies. Specifically, we focus on a end-to-end
deep learning based real-time control system,
which was developed for a real self-driving car, NVIDIA
DAVE-2~\cite{Bojarski2016}, and use the same methodology on a
smaller and less costly setup. In developing the testbed, our
goals are (1) to analyze real-time issues in DNN based end-to-end
control; and (2) to evaluate real-time performance of contemporary embedded
platforms for such workload.

% DeepPicar introduction
In this paper, we present DeepPicar, a low-cost autonomous car
platform for research. From a hardware perspective,
DeepPicar is comprised of a Raspberry Pi 3 Model B quad-core
computer, a web camera and a RC car, all of which are affordable
components (less than \$100 in total).
The DeepPicar, however, employs state-of-the-art AI
technologies, including a vision-based end-to-end control system that
utilizes a deep convolutional neural network (CNN).
The network receives an image frame from a single forward
looking camera as input and generates a predicted steering angle
value as output at each control period in \emph{real-time}.
The network has 8 layers, about 27 million connections
and 250 thousand parameters (weights).
The network architecture is almost identical to that of NVIDIA's DAVE-2
self-driving car~\cite{Bojarski2016}, which uses a much more powerful
computer (Drive PX computer~\cite{drivepx}) than a Raspberry Pi~3.
We chose to use a Pi 3 not only because it is affordable, but also because it is representative
of today's mainstream low-end embedded multicore platforms found in
smartphones and other embedded devices.

%% Other than the difference in scale (RC car vs. real car), the only other
%% differences between the two systems---from the computing
%% perspective---are that our system is implemented in
%% TensorFlow~\cite{abadi2016tensorflow} and runs on a Raspberry Pi 3
%% whereas NVIDIA's DAVE-2 systems is implemented in Torch
%% 7~\cite{collobert2011torch7} and runs on a Drive PX computer (NVIDIA's
%% automotive specialized computing system~\cite{drivepx}), which is more
%% powerful but also more expensive.

% how we trained (maybe moved to a later section)
We apply a standard imitation learning methodology to train the car to
follow tracks on the ground. We collect data for
training and validation by manually
controlling the RC car and recording the vision (from the webcam
mounted on the RC-car) and the human control inputs. We then train the
network offline using the collected data on a desktop computer, which
is equipped with a NVIDIA GTX 1060 GPU. Finally, the trained network is copied
back to the Raspberry Pi, which is then used to perform inference
operations---locally on the Pi---in the car's main control loop in
real-time. For real-time control, each inference operation must
be completed within the desired control period (e.g., 33.$\overline{\mbox{33}}$ ms
 period for 30 Hz control frequency).
% how we evaluated (in terms real-time performance)

% what are our findings?
Using the DeepPicar platform, we systematically analyze its real-time
capabilities in the context of deep-learning based real-time
control, especially on real-time deep neural network inferencing.
We also evaluated other, more powerful, embedded computing
platforms to better understand achievable real-time performance of
DeepPicar's deep-learning based control system and the impact of
computing hardware architectures.

% Findings:
%
% - we find DNN processing is an ideal real-time workload from the
%   perspective of timing predictability as the amount of computation
%   needed is fixed at design time and doesn't change over input.
%
% - real-time processing of non-trivial DNN inferencing is feasible
%   on today's embedded multicore platform, even as inexpensive as
%   raspberry pi3.
%
% - all tested platforms have enough computing capacity to consolidate 
%   multiple DeepPicar's CNN workloads or other compute intensive tasks
%   simultaneously.
%
% - however, when consolidating multiple workloads shared resource
%   contention is an important issue. we carefully analyzed
%   sensitivity of DeepPicar's workload with respect to shared
%   hardware resources.
%
% - in particular, we find DNN processing is highly sensitive to
%   memory performance. performance severely affected when memory
%   bandwidth is contented.
%
% - however, we find that dnn processing is not sensitive to cache
%   space. we find applying cache partitioning does not help improve
%   isolation of DNN workload when memory bandwidth is contended.
From the systematic study, we made a number of interesting
observations from the perspective of real-time systems.
% predictability
First, we find that DNN inferencing is, in a sense, an ideal real-time
workload as the amount of computation needed to compete a single
inference is fixed at the DNN architecture design time and does not
change at runtime over different inputs (e.g., different image frame).
This means that it is feasible to achieve highly predictable and
tightly bounded worst-case execution time estimation.

% performance on low-cost embedded multicore
Second, we find that real-time processing of DeepPicar's DNN-based
control loop is feasible on today's embedded computing platforms, even 
as inexpensive as the Raspberry Pi 3. We find that the control loop 
completes in under 33.$\overline{\mbox{33}}$ ms, or 30 Hz, using just two 
cores of the Raspberry Pi 3's quad-core CPU (w/o using its GPU), and can
do so 100\% of the time; we can achieve 20 Hz control frequency using
just one core of the Raspberry Pi 3's quad-core CPU.
Other tested embedded platforms, Intel UP and NVIDIA TX2, offer even
better performance, and are capable of supporting deep-learning based 
real-time control over 100 Hz control frequency on the TX2 when its
embedded GPU is used.
We find that all the tested embedded computing platforms have 
enough computing capacity to consolidate multiple instances of
DeepPicar's DNN workload.

% all platforms have enough capacity to consolidate
Third, contention in shared hardware resources remains an important
issue when consolidating multiple workloads. 
In particular, we find that DNN inferencing is highly sensitive to
memory performance inference as we observe up to an 11.6X control loop
execution time increase, when memory performance intensive
applications are co-scheduled on idle CPU cores.
We find, however, that DNN inferencing is not sensitive to cache
space; we show that applying cache partitioning has negligible impact
in providing timing isolation on consolidated scenarios. This finding
suggests that in order to guarantee real-time performance of DNN-based
control applications, solutions to guarantee its memory performance
are much more important.

% contributions
The {\bf contributions} of this paper are as follows:
\begin{itemize}
  \item We provide the design and implementation of a low-cost
    autonomous car testbed, DeepPicar, which is based on a
    state-of-the-art end-to-end deep learning, as
    open-source~\footnote{The source code, build 
instruction the DeepPicar
can be found at: \url{https://github.com/heechul/picar}}, with the
    goal of lowering the economic and safety-related barriers to study
    autonomous cars.

  \item We systematically analyze the characteristics of DeepPicar's
    DNN inferencing workload on a number of representative
    contemporary embedded computing platforms from the perspective of
    real-time systems. In particular, to the best of our knowledge,
    our work is first to investigate the impacts of shared resource
    contention on DNN inferencing workloads in the context of
    real-time systems.

\end{itemize}

The remaining sections of the paper are as follows: Section II 
provides a background of applications in autonomous driving and related works.
 Section III gives an overview of the DeepPicar platform, including the 
high-level system and the methods used for training and inference. 
Section IV presents our evaluation of the platform and how different 
factors can affect performance. Section V gives a comparison between 
the Raspberry Pi 3 and other embedded computing platforms to 
determine their suitably for autonomous driving research. The paper finishes with 
conclusions in Section VI.
